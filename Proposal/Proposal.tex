\documentclass[12pt]{article}
\usepackage[margin=1in]{geometry}
\usepackage{amsmath}
\usepackage{amsthm}
\usepackage{float}
\usepackage{graphicx}
\usepackage{natbib}
\usepackage{enumitem}
\usepackage{booktabs}
\usepackage{hyperref}
\usepackage[utf8]{inputenc}
\usepackage{lipsum}
\usepackage[english]{babel}
\usepackage[autostyle, english = american]{csquotes}
\MakeOuterQuote{"}
\title{My Statistics 3494W Proposal}

\author{Michael  Marcaccio}

\date{October 4 2022}

\begin{document}

\maketitle

\section*{Introduction}
Call centers are a staple of the finanical industry with agents working in the United States, Europe, Asia and Africa \citep{ibrahim2016modeling}. 
Effective management is essential to running call centers and businesses are consistently trying to model and optimize when customers are calling with questions. 
\citet*{evensen1999effective} describes the recent growth of call centers as being, "viewed as little more than lower cost channels for customer
problem resolution, are quickly becoming a powerful means of service delivery with a potential for substantial revenue generation".
I have chosen this to be my topic of study as I have a strong interest in working in the finanical industry after graduation and I have an unique opportunity to utilize real-world data from an undisclosed
finanical firm. Due to this, My findings may provide insight that would benefit the company and its members.

\section*{Specific Aims}
I will be attempting to model call volume per a 15 minute interval and determining if there is any seasonality in the calls. With each interval,
I will find the optimal number of employees needed to handle the amount of incoming calls based on typical call length. This is extremely important
to be able to forecast call volume and be optimally prepared with a proper number of employees for anticipated spikes in calls. If a business is not prepared
this will create a increase in holdtime which will result in members becoming frustrated and discouraged. \citet*{ibrahim2016modeling} states, "The accurate modeling and forecasting of future call arrival volumes is a 
complicated issue which is critical for making important operational decisions, such as staffing and scheduling, in the call center".
    
\section*{Data Description}
I wrote SQL queries to obtain my data sets from the undisclosed finanical firm which comprise of the call center agents performance and the interactions between members. 
The data goes back Feburary 3, 2020 to October 3, 2022. There is 466,566 observations with their respective Call Date, Caller ID, Interaction Outcome,
Reason for calling, Wait Time, Hold Time, Talk Duration, Result of Call, and Agent ID among other things. To protect privacy, each Caller ID
and Agent ID has been encrypted. There is a select code of symbols for Caller ID and Agent ID has been reduced to Agent 1,2,3 ... instead of 
their names.

\bibliographystyle{chicago}
\bibliography{citations.bib}
\end{document}