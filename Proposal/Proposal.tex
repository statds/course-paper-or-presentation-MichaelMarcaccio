\documentclass[12pt]{article}
\usepackage[margin=1in]{geometry}
\usepackage{amsmath}
\usepackage{amsthm}
\usepackage{float}
\usepackage{graphicx}
\usepackage{natbib}
\usepackage{enumitem}
\usepackage{booktabs}
\usepackage{hyperref}
\usepackage[utf8]{inputenc}
\usepackage{lipsum}
\usepackage[english]{babel}
\usepackage[autostyle, english = american]{csquotes}
\MakeOuterQuote{"}
\title{My Statistics 3494W Proposal}

\author{Michael  Marcaccio}

\date{October 4 2022}

\begin{document}

\maketitle

\section*{Introduction}
Call centers are a staple of the finanical industry with agents working in the United States, Europe, Asia and Africa \citep{ibrahim2016modeling}. 
Effective management is essential to running call centers and businesses are consistently trying to model and optimize when customers are calling with questions. 
\citet*{evensen1999effective} describes the recent growth of call centers as being, "viewed as little more than lower cost channels for customer
problem resolution, are quickly becoming a powerful means of service delivery with a potential for substantial revenue generation".
I have chosen this to be my topic of study as I have a strong interest in working in the finanical industry after graduation and I have an unique opportunity to utilize real-world data from an undisclosed
finanical firm. Due to this, My findings may provide insight that would benefit the company and its members.

\section*{Specific Aims}
Research Question: Is there any seasonality in call center data per 15 minute intervals and what is the optimal number of employees needed per 30 minute intervals?
I will be attempting to model call volume per a 15 minute interval and determining if there is any seasonality in the calls. With each 30 minute interval,
I will find the optimal number of employees needed to handle the amount of incoming calls based on typical call length. This is extremely important
to be able to forecast call volume and be optimally prepared with a proper number of employees for anticipated spikes in calls. If a business is not prepared
this will create a increase in holdtime which will result in members becoming frustrated and discouraged. \citet*{ibrahim2016modeling} states, "The accurate modeling and forecasting of future call arrival volumes is a 
complicated issue which is critical for making important operational decisions, such as staffing and scheduling, in the call center".
    
\section*{Data Description}
I wrote SQL queries to obtain my data sets from the undisclosed finanical firm which comprise of the call center agents performance and the interactions between members. 
The data goes back Feburary 3, 2020 to October 3, 2022. There is 466,566 observations with their respective Call Date, Caller ID, Interaction Outcome,
Reason for calling, Wait Time, Hold Time, Talk Duration, Result of Call, and Agent ID among other things. To protect privacy, each Caller ID
and Agent ID has been encrypted. There is a select code of symbols for Caller ID and Agent ID has been reduced to Agent 1,2,3 ... instead of 
their names.

\section*{Research Design}
I will be using Rstudio to create a model to forecast the call center data. I plan on doing research on previous ways call center data
has been forecast before. I am planning on creating multiple models based on a 15 minute interval using liner regression, poisson distribution,
and autoregressive integrated moving average (ARIMA) models \citep*{ibrahim2016modeling}. By creating figures I can
easily describe the patterns to my audience. These methods can clearly show a pattern which I can use to back up my claims on how many 
employees should be present at certain times. Once I can find a clear distribution to describe the call center patterns, I will be
able to possibly going into further detail describing if there is a relationship between call length and call reason, and if all agents are
performing at the same level. With these methods, I believe I can investigate and answer the research question. 

\section*{Discussion}
I expect to see in an increase in calls during the early afternoon. This is typically when people have a lunch break and would have 
easy access to call the call center. There may be in increase in the early evening after people get out of work. In regards to agent
performance, I feel that most agents will be relatively consistent in regards to number of calls, wait time, and other factors. I believe 
this finanical firm would have the best of the best working for them. Because I have data from 2020, that may skew the data due to the impacts
of COVID-19. My impacts will allow this finanical firm to optimize their call center and have a more efficient and effective process. They will
also have a better expectation of what will happen after something unexpected occurs, like a website outage. If my investigation is not what I 
expect, I believe that would be okay because this would give a different prospectives of when and why people call. I believe the more prospective
the more efficient the call center can be. My findings have the ability to both challenge and corroborate assumptions when it comes to the best way to 
forecast call center data depending on my conclusions.


\section*{Conclusion}
Overall, my research can provide excellent insight into forecasting call center data. I can help finanical firms with ways to deal with spikes in 
calls and how to be efficient during the least amount of activity throughout the day. I will set a new standard on how agents should be operating to set
a baseline on performance. With the amount of data at my hands, there is also an opportunity for others to collaborate and extend this analysis after this semester.
\bibliographystyle{chicago}
\bibliography{citations.bib}
\end{document}